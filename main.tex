\documentclass[twoside,twocolumn]{article}
\usepackage[utf8]{inputenc}
\usepackage[T2A]{fontenc}						% кодировка
\usepackage[english,russian]{babel}				% локализация и переносы
\usepackage{color}
%\usepackage[sc]{mathpazo} 					% Use the Palatino font
%\linespread{1.05} 							% Line spacing - Palatino needs more space between lines
%\usepackage{microtype} 					% Slightly tweak font spacing for aesthetics
\usepackage[hmarginratio=1:1,top=32mm,columnsep=20pt]{geometry} 	% Document margins

\usepackage{fancyhdr} 					% Headers and footers
\pagestyle{fancy} 						% All pages have headers and footers
\fancyhead{} 							% Blank out the default header
\fancyfoot{} 							% Blank out the default footer
\fancyhead[C]{С днем рождения, Дракула! $\bullet$ Автор Джонатан Симс (Jonathan Sims)}
\fancyfoot[RO,LE]{\thepage} 			% Custom footer text

\usepackage{titlesec} 					% Allows customization of titles
\renewcommand\thesection{\Roman{section}} % Roman numerals for the sections
\renewcommand\thesubsection{\roman{subsection}} % roman numerals for subsections
\titleformat{\section}[block]{\large\scshape\centering}{\thesection.}{1em}{} % Change the look of the section titles
\titleformat{\subsection}[block]{\large}{\thesubsection.}{1em}{} % Change the look of the section titles

\usepackage[hang, small,labelfont=bf,up,textfont=it,up]{caption} % Custom captions under/above floats in tables or figures
\usepackage{booktabs} 					% Horizontal rules in tables

\usepackage{enumitem} 					% Customized lists
\setlist[itemize]{noitemsep} 			% Make itemize lists more compact

\usepackage{titling} 					% Customizing the title section

\usepackage{abstract} 											% Allows abstract customization
%\renewcommand{\abstractnamefont}{\normalfont\bfseries} 			% Set the "Abstract" text to bold
\renewcommand{\abstracttextfont}{\normalfont\small\itshape} 	% Set the abstract itself to small italic text

%----------------------------------------------------------------------------------------
%	TITLE SECTION
%----------------------------------------------------------------------------------------

\setlength{\droptitle}{-4\baselineskip} % Move the title up
\pretitle{\begin{center}\Huge\bfseries} % Article title formatting
\posttitle{\end{center}} 				% Article title closing formatting
\title{С днем рождения, Дракула!} % Article title
% \author{Элевтер Араней}					% Your name
\author{\textsc{Чудовищно веселая игра для 3 - 6 ужасающих созданий ночи.}\thanks{Happy Birthday Dracula! by Jonathan Sims} \\[1ex] \\ % Your institution
\normalsize {Элевтер Араней и конвент Мастеров "Львы и Драконы"}
}
\date{} 								% Leave empty to omit a date. Я убрал \today
\renewcommand{\maketitlehookd}{%
\begin{abstract}
\noindent "С днём рождения, Дракула!" - это ролевая игра для трех - шести игроков, в которой вы - монстр, что пытается убедить людей отправиться с ним на праздник к Дракуле.
Для игры вам нужны колода игральных карт, один шестигранный кубик и несколько друзей.
Обычная продолжительность игры от 40 до 120 минут в зависимости от числа игроков.
\end{abstract}
}

%-----------------
\begin{document}
\renewcommand{\abstractname}{\vspace{-\baselineskip}}

\maketitle

Ура! Завтра день рождения Дракулы, и в первый раз за последние 120 лет вас наконец-то пригласили на празднество!

Это ваш шанс познакомиться со всеми жутко классными монстрами, ходят слухи, что даже Всадник без головы может показаться, а он такой романтичный...

Вам необходимо произвести хорошее впечатление! К сожалению, приглашение несколько туманно насчет деталей празднования.
В нем не говориться какого это рода празднование и сколько гостей хозяин ждет, что вы приведете с собой.

Ну и пусть, это ведь Дракула, а он всегда жаждет обильного празднества на случай, если проголодается.
Ко всему прочему, это так важно выглядеть популярным.

Но времени не так много, так что лучше скорее нырнуть в сумрак ночи и найти как можно больше людей, чтобы привести их с собой.
Будем надеяться, что Дракула не съест их ВСЕХ.

\section*{Фазы игры}
Каждая игра проходит в три фазы:
\begin{itemize}
   \item Фаза 1 - Создание своего монстра
   \item Фаза 2 - Подружитесь с людьми
   \item Фаза 3 - Время праздновать!
\end{itemize}

\section*{Фаза 1 - Создание своего монстра}
Игрок, чей день рождения наступит скорее других действует первым.

Если у двух или большего числа игроков день рождения в один день, то они должны изобразить Дракулу и тот, у кого это получилось лучше, чем у других ходит первым.

Начиная с первого игрока и по часовой стрелке каждый игрок бросает кубик и в "Таблице 1: Виды монстров"\  находит к какому из видов монстров он принадлежит. Затем вытяните одну карту и по "Таблице 3: Бзики и Причуды"\ найдите каков ваш монстр сам по себе. Это поможет определиться с тем, как ваш монстр приступит к убеждению людей отправиться на праздник с вами.

Желательно дать имя своему монстру, но совсем не обязательно.

\subsection*{Например:}
{\itshape
Это день рождения Маши, поэтому она первый игрок. Она выбросила на кубике 4 и вытянула 8 треф. Это значит, что она Мумия и Владелица совы.

Она довольна этим и решает назвать свою мумию Пыльный Шура. Также она решает назвать сову У-ухом Третьим, причем Сова тоже вся в бинтах.

Никто не в силах ей помешать.}

\begin{table}
\caption{Виды монстров}
\centering
\begin{tabular}{cl}
\toprule
% \multicolumn{2}{c}{Название} \\
\cmidrule(r){1-2}
На кубике & Вид \\
\midrule
1 & Вампир \\
2 & Франкенштейн \\
3 & Оборотень \\
4 & Мумия \\
5 & Призрак \\
6 & Умертвие \\
\bottomrule
\end{tabular}
\end{table}

\section*{Фаза 2 - Подружитесь с людьми}
Начиная с первого игрока, каждый по очереди берет на себя роль человека, с которым нужно подружиться (можно сказать, вероятного друга, но 100\% человека).

Игрок, изображающий человека, не может сделать его другом своего монстра, потому что тот, предположительно, ушел за молочным коктейлем или практикуется в новых па своего танца в этот ход.

Игрок берет две карты и в соответствии с "Таблицей 3: Бзики и Причуды"\ узнает каков его человек, а затем объявляет это остальным.
Затем, начиная с игрока по левую руку, каждый получает 90 секунд на то, чтобы убедить этого человека отправиться с ним на празднование дня рождения Дракулы.
Человек, в свою очередь, может задавать монстру вопросы, отвечать на его предложения и просто поддерживать разговор, однозначно, что нельзя допустить ужасающего монолога.

Ограничение в 90 секунд лишь общая рекомендация, человек волен оборвать разговор в любой момент, судя по тому, как хорошо или плохо идут дела.
Разговор может длиться и дольше, но имейте в виду, что не очень хорошо давать одному игроку времени больше, чем другим, если вы дорожите дружеской атмосферой.
Далее по часовой стрелке каждый игрок по очереди пытается убедить человека от имени своего монстра пока не будут услышаны все предложения.

После этого человек выбирает монстра, который по его мнению больше всего подходит ему, как спутник для празднества и отдает этому игроку свою карту Бзика и Причуды.
Человек не может отказаться идти на празднование, в конце концов это ведь сам Дракула, как вы потом в глаза другим смотреть будете?

После того, как каждый из игроков побывал в роли человека, порядок хода пойдет против часовой стрелки до тех пор, пока каждый не побывает в роли человека два раза.

\subsection*{Вариант правил: Обмен монстрами}
После того, как каждый игрок побывал в роли человека, все могут изменить своего монстра на другого, вновь бросая кубик и вытягивая новую карту с Бзиком и Причудой.
Это правило мы советуем использовать, если вы обнаружили, что повторяетесь в попытках подружиться с людьми и хотите попробовать что-то новое.

Мы предполагаем, что ваш второй монстр выступает партнером первого и делит с ним свой результат по очкам, если вы не договоритесь сыграть это иначе.
Это совершенно в порядке даже если вы пытались влюбить или соблазнить человека - монстры очень открыты к самым нестандартным отношениям.

\subsection*{Например:}
{\itshape 
Сейчас очередь Артура играть за человека и он вытягивает Туз бубен и 8 червей: он Отставной охотник на монстров и Переполнен энтузиазмом по отношению ко всему происходящему.

Маша решает, что Пыльный Шура попробует подружиться с ним, потому что она предполагает, что он решил уйти в отставку после того, как обнаружил, что и у монстров тоже есть чувства, но она не угадала (он даже выстрелил гарпуном в У-уха Третьего!). 

Клара с её Тревожно беспокойным (2 пик) Франкенштейном попробует убедить Охотника вновь взять в прицел Дракулу, единственного монстра, которого Охотник так и не сделал своим трофеем. Лишь бы только затащить его на праздник, не так ли?

Люда рассчитывает, что ее Очаровывающий собеседник (4 бубен) Призрак напомнить Охотнику о всех друзьях, которых он потерял, и что всех их можно будет встретить вновь, танцующими на празднике!

- Подскажите, а Отрубленная голова Ван Хелсинга там будет? - спрашивает Отставной охотник с нескрываемым предвкушением.

- Как знать, может случиться!

Артур решает, что как бы ни было соблазнительно вновь начать охоту за Дракулой, встреча со старыми друзьями для него важней и отдает Люде Туз бубен и 8 червей.

После того, как все сделали свой ход, Маша решает, что гарпун в любимую Сову - это слишком, и она перебрасывает кубик и вытягивает карту для создания нового монстра. Она снова мумия, но в этот раз Приходящая в восторг от всего зловещего (5 пик). Ей только остается со вздохом признать, что новой жене Пыльного Шуры, скорее всего, повезет не больше чем раньше...
}

\section*{Фаза 3 - Время праздновать!}
Пришло время всем гостям и монстрам отправиться на празднование к Дракуле. Только в этот момент монстры могут узнать какого рода это празднество и может случиться, что некоторые из гостей будут приняты лучше, чем другие.

Первый игрок бросает шестигранный кубик и в "Таблице 2: Какое сегодня празднество?"\ определяет, что задумал Дракула для своих гостей. Далее все игроки складывают очки, которые им принесли карты гостей, пропуская карту Бзика и Причуды своего монстра.

\begin{table}
\caption{Какое сегодня празднество?}
\centering
\begin{tabular}{cl}
\toprule
% \multicolumn{2}{c}{Название} \\
\cmidrule(r){1-2}
На кубике & Празднество \\
\midrule
1 & Кровавый пир \\
2 & Роскошный банкет \\
3 & Зловещий маскарад \\
4 & Забойная дискотека \\
5 & Званый ужин \\
6 & Теплый вечер с близкими \\
\bottomrule
\end{tabular}
\end{table}

\begin{quote}
ВАЖНО

Если кто-нибудь из игроков в любой момент игры отпустит комментарий в духе, мол, это не день рождения Дракулы, а день умерщвления или оживления, и все в таком духе, он немедленно теряет 5 очков.

Дракула терпеть НЕ МОЖЕТ педантов.
Даже, если вы сказали это, когда были в роли гостя, а не своего монстра.
Дракуле известно, что это были вы.

\end{quote}

Игрок с наибольшим числом очков побеждает и становится Лучшим Другом Дракулы (в этом году).

Если среди игроков ничья, то побеждают оба, потому что Дракула умеет дружить.

Если вас не устраивает такая победа при равенстве очков, мы рекомендуем вам переосмылить ваши жизненные приоритеты.

\subsection*{Какие бывают празднества?}
\subsection{Кровавый пир}
Ой, похоже, что графа не интересуют беседы и танцы, он жаждет крови!

Если вы не исключительно аппетитны, то сегодня речь идет больше о количестве, чем о качестве человеческих гостей. Все карты гостей приносят по 1 победному очку, а джокеры целых 5 очков.

\subsection{Роскошный банкет}
Дракула дает один из своих знаменитых балов. Мы говорим о сорока семи нежнейших окороках подаваемых на серебряных подносах и столько старого марочного вина, сколько вы сможете выпить.

Гости из высшего общества ценятся очень высоко, хотя самых скучных из них могут попросить удалиться.

Пики - 3 очка. Бубны и джокеры - 2 очка.

Трефы - 1 очко. Черви - 0 очков.

\subsection{Зловещий маскарад}
Дракула демонстрирует свою пугающую сущность. Гротескные маски обязательны для всех, кто еще не выглядит в достаточной степени пугающе.

Общая атмосфера зловещего веселья, а тем, кто слишком напыщен, чтобы принимать в этом участие, совсем не рады.

Трефы - 3 очка. Черви и джокеры - 2 очка.

Бубны - 1 очко. Пики - 0 очков.

\subsection{Забойная дискотека}
В этом году Драк хочет потанцевать! Под угарный ритм и улётные мелодии! Полная чума!

Всех, кто горит желанием зажечь, встречают с распростертыми объятиями, и совсем нет времени на разговоры, когда так рубят басы.

Черви - 3 очка. Трефы и джокеры - 2 очка.

Пики - 1 очко. Бубны - 0 очков.

\subsection{Званый ужин}
Как же все культурно и пристойно! Граф организовал умиротворенный званый ужин с изобилием сыра и портвейна (по крайней мере вам кажется, что это портвейн) после застолья.

Интересный и эрудированные гости тепло приветствуются, а всякие чудаки чувствую себя явно не в своей тарелке.

Бубны - 3 очка. Пики и джокеры - 2 очка.

Черви - 1 очко. Трефы - 0 очков.

\subsection{Теплый вечер с близкими}
Ой, похоже что Драк надеялся на спокойный ужин со своими друзьями. Он вежлив со всем, кого вы притащили с собой, но вы уверенны, что он с удовольствием обошелся бы без них.

Все карты гостей минус 1 очко.

\subsection*{Например:}
{\itshape 
Скоро день рождения Маши и она выбросила на кубике 5 - похоже, что у Дракулы будет званый ужин.

Артур смог привести с собой Всемирно известного путешественника (9 бубен - 3 очка) с Очень громким голосом (6 червей - 1 очко) и Одержимого внешним видом (3 пик - 2 очка) Епископа (Красный джокер - 2 очка), итого 8 очков.

Единственный гость Маши был Почти глухой (Валет червей - 1 очко), но с Роскошными усами (8 пик - 2 очка), итого 3 очка.

Люда хорошо справилась и привела трех гостей. Среди которых были Переполненный энтузиазмом (8 червей - 1 очко) Отставной охотник на монстров (Туз бубен - 3 очка), и еще Безупречно вежливый (4 пик - 2 очка) и Глубоко-глубоко чувствующий музыку (7 червей - 1 очко). К сожалению ее третий гость был Страшен как смертный грех (Туз треф - 0 очков) и Полный гот (Дама треф - 0 очков), итого она смогла набрать только 7 очков.

Клара пришла с двумя гостями, хотя и не очень рада первому из них, который был Исключительно голоден (5 пик - 2 очка) и Ворчливым без остановки (7 треф - 0 очков).

Но ее второй гость был Профессор чего-то странного (5 бубен - 3 очка), Маша сообщила, что это знаток истории чулок и носок, ко всему прочему он был Самодовольный и Высокомерный (6 пик - 2 очка), так маша набрала 7 очков и разделила второе место с Людой.

Похоже, что лучшим другом Дракулы стал Артур! Но самое важное, что всем было весело.
}

\section*{Таблица 3: Бзики и Причуды}
Смотреть на 5 странице.
%----------------------------------------------------------------------------------------
%	REFERENCE LIST
%----------------------------------------------------------------------------------------

\begin{thebibliography}{99} % Bibliography - this is intentionally simple in this template

\bibitem[Happy Birthday Dracula!]{Link:2016}
\newblock {\em by Jonathan Sims},
\newblock jonathan-sims.com 

\end{thebibliography}

%----------------------------------------------------------------------------------------

\pagebreak

\begin{table}[!htb]
    \caption{Бзики и Причуды}
    \begin{minipage}{.5\linewidth}
      \caption*{Пики}
      \centering
        \begin{tabular}{ll}
            Туз & Настоящий Граф\\
2 & Такой Декадент и Эстет\\
3 & Одержимый внешним видом\\
4 & Безупречно вежливый\\
5 & Исключительно голодный\\
6 & Самодовольный и Высокомерный\\
7 & Курит старинную трубку\\
8 & Пышные усы\\
9 & Старый. Такой СОВСЕМ старый-старый\\
10 & Напыщенный и Болтливый\\
Валет & Придирчивый гурман\\
Дама & Говорит, что лично знаком с Королевой\\
Король & Безумно богат
        \end{tabular}
    \end{minipage}%
\begin{minipage}{2.7\linewidth}
      \centering
        \caption*{Трефы}
        \begin{tabular}{ll}
Туз & Страшен как смертный грех\\
2 & Тревожно беспокойный\\
3 & Тревожный вплоть до паранойи\\
4 & Какой-то зловещий\\
5 & Приходящий в восторг от всего зловещего\\
6 & Пытающийся скрыть убийство\\
7 & Ворчливый без остановки\\
8 & Владелец/владелица совы\\
9 & Сумасшедший ученый\\
10 & Без причин драматизирующий\\
Валет & Ведьма старой закваски\\
Дама & Полный гот\\
Король & Гробовщик\\
        \end{tabular}
   \end{minipage}
\hfill   
\vfill
\vfill
\hfill

\begin{minipage}{.5\linewidth}
      \caption*{Черви}
      \centering
        \begin{tabular}{ll}
Туз & Отличный танцор\\
2 & Покрытый блестками\\
3 & Одет весьма легкомысленно\\
4 & Уже пьян\\
5 & В одежде самой павлиньей раскраски\\
6 & С очень громким голосом\\
7 & Глубоко-глубоко чувствующий музыку\\
8 & Переполненный энтузиазмом\\
9 & Балдеющий от монстров\\
10 & Из своей крутой тусовки\\
Валет & Почти глухой/слепой\\
Дама & Возбужденный и энергичный тусовщик\\
Король & Диджей с модными хитами
        \end{tabular}
    \end{minipage}%
\begin{minipage}{2.7\linewidth}
      \caption*{Бубны}
      \centering
        \begin{tabular}{ll}
Туз & Отставной охотник на монстров\\
2 & Консерваторский музыкант\\
3 & Доктор медицинских наук\\
4 & Очаровывающий собеседник\\
5 & Профессор чего-то очень странного\\
6 & Напыщенный с вычурными манерами актер\\
7 & С приятным глубоким голосом\\
8 & Восхищающийся совершенно всем\\
9 & Всемирно известный путешественник\\
10 & Мечтающий всем нравиться\\
Валет & Шекспировский актер\\
Дама & Фанат Дракулы\\
Король & Вчерашняя знаменитость
        \end{tabular}
    \end{minipage}%
\end{table}

\noindent Красный джокер: Викарий/Епископ

\noindent Черный джокер: Весь такой полнокровный!

\end{document}
